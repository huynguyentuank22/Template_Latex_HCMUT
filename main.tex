% include package
\documentclass[a4paper]{article}
\usepackage{listings}
\usepackage{algorithm}
\usepackage[noend]{algpseudocode}
\usepackage{amsmath}
\usepackage[table]{xcolor}
\usepackage{xcolor}
\usepackage[utf8]{inputenc}
\usepackage{csquotes}
\usepackage{tcolorbox}
\usepackage{verbatim}
\usepackage{a4wide,amssymb,epsfig,latexsym,multicol,array,hhline,fancyhdr}
\usepackage{vntex}
\usepackage{amsmath}
\usepackage{lastpage}
\usepackage[lined,boxed,commentsnumbered]{algorithm2e}
\usepackage{enumerate}
\usepackage{color}
\usepackage{graphicx}							% Standard graphics package
\usepackage{array}
\usepackage{tabularx, caption}
\usepackage{multirow}
\usepackage{multicol}
\usepackage{rotating}
\usepackage{graphics}
\usepackage{geometry}
\usepackage{setspace}
\usepackage{epsfig}
\usepackage{tikz}
\usetikzlibrary{arrows,snakes,backgrounds}
\usepackage{hyperref}
\usepackage{indentfirst}
\hypersetup{urlcolor=blue,linkcolor=black,citecolor=black,colorlinks=true} 
%\usepackage{pstcol} 								
% PSTricks with the standard color package
% Định nghĩa các màu cho syntax highlighting
\definecolor{codegreen}{rgb}{0,0.6,0}
\definecolor{codegray}{rgb}{0.5,0.5,0.5}
\definecolor{codepurple}{rgb}{0.58,0,0.82}
\definecolor{backcolour}{rgb}{0.95,0.95,0.92}

% Định nghĩa cài đặt cho hiển thị code
\lstdefinestyle{mystyle}{
    backgroundcolor=\color{backcolour},
    commentstyle=\color{codegreen},
    keywordstyle=\color{blue},
    numberstyle=\tiny\color{codegray},
    stringstyle=\color{codepurple},
    basicstyle=\footnotesize\ttfamily,
    breakatwhitespace=false,
    breaklines=true,
    captionpos=b,
    keepspaces=true,
    numbers=left,
    numbersep=5pt,
    showspaces=false,
    showstringspaces=false,
    showtabs=false,
    tabsize=2
}

\lstset{style=mystyle}

\newtheorem{theorem}{{\bf Theorem}}
\newtheorem{property}{{\bf Property}}
\newtheorem{proposition}{{\bf Proposition}}
\newtheorem{corollary}[proposition]{{\bf Corollary}}
\newtheorem{lemma}[proposition]{{\bf Lemma}}

\AtBeginDocument{\renewcommand*\contentsname{Mục lục}}
\AtBeginDocument{\renewcommand*\refname{Tài liệu tham khảo}}
%\usepackage{fancyhdr}
\setlength{\headheight}{40pt}
\pagestyle{fancy}
\fancyhead{} % clear all header fields
\fancyhead[L]{
 \begin{tabular}{rl}
    \begin{picture}(25,15)(0,0)
    \put(0,-8){\includegraphics[width=8mm, height=8mm]{Image/hcmut.png}}
    %\put(0,-8){\epsfig{width=10mm,figure=hcmut.eps}}
   \end{picture}&
	%\includegraphics[width=8mm, height=8mm]{hcmut.png} & %
	\begin{tabular}{l}
		\textbf{\bf \ttfamily TRƯỜNG ĐẠI HỌC BÁCH KHOA - ĐHQG-HCM}\\
		\textbf{\bf \ttfamily KHOA KHOA HỌC VÀ KĨ THUẬT MÁY TÍNH}
	\end{tabular} 	
 \end{tabular}
}
\fancyhead[R]{
	\begin{tabular}{l}
		\tiny \bf \\
		\tiny \bf 
	\end{tabular}  }
\fancyfoot{} % clear all footer fields
\fancyfoot[L]{\scriptsize \ttfamily Bài tập lớn Lập trình nâng cao - HK 2 năm học 2023 - 2024}
\fancyfoot[R]{\scriptsize \ttfamily Trang {\thepage}/\pageref{LastPage}}
\renewcommand{\headrulewidth}{0.3pt}
\renewcommand{\footrulewidth}{0.3pt}
\renewcommand{\baselinestretch}{1.5}

%%%
\setcounter{secnumdepth}{4}
\setcounter{tocdepth}{3}
\makeatletter
\newcounter {subsubsubsection}[subsubsection]
\renewcommand\thesubsubsubsection{\thesubsubsection .\@alph\c@subsubsubsection}
\newcommand\subsubsubsection{\@startsection{subsubsubsection}{4}{\z@}%
                                     {-3.25ex\@plus -1ex \@minus -.2ex}%
                                     {1.5ex \@plus .2ex}%
                                     {\normalfont\normalsize\bfseries}}
\newcommand*\l@subsubsubsection{\@dottedtocline{3}{10.0em}{4.1em}}
\newcommand*{\subsubsubsectionmark}[1]{}
\makeatother

\begin{document}

\begin{titlepage}
\begin{center}
ĐẠI HỌC QUỐC GIA THÀNH PHỐ HỒ CHÍ MINH \\
TRƯỜNG ĐẠI HỌC BÁCH KHOA \\
KHOA KHOA HỌC VÀ KĨ THUẬT MÁY TÍNH
\end{center}

\vspace{1cm}

\begin{figure}[h!]
\begin{center}
\includegraphics[width=3cm]{Image/hcmut.png}
\end{center}
\end{figure}

\vspace{1cm}


\begin{center}
\begin{tabular}{c}
\multicolumn{1}{l}{\textbf{{\Large ADVANCED PROGRAMMING (CO2039)}}}\\
~~\\
\hline
\\
%\multicolumn{1}{l}{\textbf{{\Large Bài tập lớn}}}\\
\\
\textbf{{\Huge ADVANCED PROGRAMMING}}\\
\textbf{{\Huge ASSIGNMENT - HK232}}\\
\\
\\
\hline
\end{tabular}
\end{center}

\vspace{2.5cm}

\begin{table}[h]
	\begin{tabular}{rrlr}
		\hspace{3.5 cm} & Giảng viên hướng dẫn: & PGS TS. Trần Văn Hoài      \\
		\\
		\\
			            & Sinh viên thực hiện: & Nguyễn Tuấn Huy & 2211253 \\
                        &                      & Nguyễn Văn Nhật Huy & 2211254 \\
                        &                      & Võ Hoàng Huy & 2211298 \\
	\end{tabular}
\end{table}

\vspace{3cm}

\begin{center}
{\footnotesize TP. HỒ CHÍ MINH, THÁNG 04/2024}
\end{center}
\end{titlepage}


%\thispagestyle{empty}

\newpage
\tableofcontents
\newpage


%%%%%%%%%%%%%%%%%%%%%%%%%%%%%%%%%
\section {Lập trình hàm (Functional Programming) trong ngôn ngữ lập trình Python}
\begin{lstlisting}[language=Go, caption=Example Go code]
package main

import "fmt"

func main() {
    fmt.Println("Hello, world!")
}
\end{lstlisting}

\section {Design Pattern}
\begin{table}[h]
    \centering
    \begin{tabular}{|p{7cm}|p{7cm}|}
        \hline
        \rowcolor{blue!10}\textbf{Lập trình Hàm} & \textbf{Lập trình Hướng đối tượng} \\
        \hline
        huhu & hehe \\
        \hline
        
        \hline
        haha & hoho  \\
        \hline
    \end{tabular}
    \caption{So sánh Functional Programming và OOP}
    \label{tab:example}
    
\end{table} 

\section{Gì đó}
\subsection{Gì gì đó}
{Hello World}

\section{Lmao}
\input{lmao}
%%%%%%%%%%%%%%%%%%%%%%%%%%%%%%%%%
\newpage
\begin{thebibliography}{80}
\bibitem{bib1}
GeeksforGeeks
\bibitem{bib2}
Coursera
\end{thebibliography}
\newpage
\end{document}
